\documentclass[main.tex]{subfiles}
\begin{document}
\chapter{图论}
	\section{最大团}
		\lstinputlisting{code/graph_theory/max_cluster.cc}
		\section{图的遍历和连通性}
        \subsection{割点和桥}
            \lstinputlisting{code/gedian.cc}
        \subsection{双连通分量}
            \lstinputlisting{code/shuangliantong.cc}
        \subsection{强连通分量}
            \lstinputlisting{code/scc.cc}
        \subsection{拓扑排序}
            \subsubsection{BFS}
            \subsubsection{判断是否成环}
            拓扑排序形成的ans的sz != n 则成环
        \subsection{2SAT}
            \lstinputlisting{code/2sat.cc}
    \section{路径}
        \subsection{非递归欧拉回路}
            \lstinputlisting{code/eulerpath.cc}
    \section{匹配}
        \subsection{二分图最大匹配}
            最小点覆盖的点数 = 二分图最大匹配 \\
            最大独立集的点数 = 总点数 - 二分图最大匹配
        \lstinputlisting{code/erfentu.cc}
        \subsection{二分图最优匹配}
            \lstinputlisting{code/km.cc}
    \section{树}
        \subsection{prim}
            \lstinputlisting{code/prim.cc}
        \subsection{曼哈顿最小距离生成树}
            \lstinputlisting{code/manhadun.cc}
    \section{网络流}
        \subsection{最大流 Dinic}
            \lstinputlisting{code/dinic.cc}
        \subsection{最大流 ISAP}
            \lstinputlisting{code/isap.cc}
        \subsection{费用流}
            \lstinputlisting{code/mcmf.cc}
        \subsection{无源无汇有容量下界网络的可行流}
            \lstinputlisting{code/yy/wuyuan.cc}
    \section{其他}
        \subsection{层次遍历}
            每次记录队列当前的sz 然后在一次循环中只出队sz次
        \subsection{图解序列}
            图解序列:一系列非负整数可以构成一个简单图的度序列 \\
            Havel定理:对于n>1,长度为n的整数序列是图解序列当且仅当d'是图解序列,d'是的删除d中最大元素Δ并将紧跟的Δ个元素减1后序列	
\end{document}


