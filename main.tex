% !TeX program = XeLaTeX
% !TeX encoding = UTF-8
\documentclass{report}
\usepackage{amsmath}
\usepackage{xcolor}
\usepackage{listings}
\usepackage{xeCJK}
\usepackage[colorlinks,linkcolor=black,anchorcolor=black,citecolor=black]{hyperref}
\usepackage[ampersand]{easylist}
\setCJKmainfont{微软雅黑}
\lstset{
    language=C++,
	basicstyle=\ttfamily,
	breaklines=true,
    numbers=left,
    numberstyle=\tiny
}
\usepackage{subfiles}
\begin{document}
\title{华理小男孩模板}
\author{
王亦凡 \\ 
\texttt{主代码手}
\and
姚远 \\ 
\texttt{队长}
\and
王泽宸 \\ 
\texttt{解题核心}
}
\maketitle
\tableofcontents
\newpage

\subfile{number_theory/index.tex}
\subfile{probability_theory/index.tex}
\chapter{数学}
    \section{矩阵}
        \subsection{矩阵类}
            \lstinputlisting{code/mat.cc}
        \subsection{高斯消元}
            \lstinputlisting{code/gauss.cc}
    \section{整除与剩余}
        \subsection{扩展欧几里得 逆元}
            \lstinputlisting{code/extgcd.cc}
        \subsection{中国剩余定理}
            \lstinputlisting{code/china.cc}
    \section{数值计算}
    \section{其他}
        \subsection{lucas定理}
            \lstinputlisting{code/lucas.cc}
        \subsection{递推求组合数}
            \lstinputlisting{code/dituic.cc}
        \subsection{单个求组合数}
            \lstinputlisting{code/single_c.cc}
        \subsection{威佐夫博弈}
            \lstinputlisting{code/wzf.cc}
        \subsection{FWT}
            \lstinputlisting{code/fwt.cc}
        \subsection{FFT}
            \lstinputlisting{code/FFT.cpp}
        \subsection{hell方程}
            $ x^2-ny^2 = 1$
            $ x[i+1] = x[1]*x[i] + n*y[1]*y[i]; $
            $ y[i+1] = x[1]*y[i] + y[1]*x[i] $
\chapter{图论}
    \section{图的遍历和连通性}
        \subsection{割点和桥}
            \lstinputlisting{code/gedian.cc}
        \subsection{双连通分量}
            \lstinputlisting{code/shuangliantong.cc}
        \subsection{强连通分量}
            \lstinputlisting{code/scc.cc}
        \subsection{拓扑排序}
            \subsubsection{BFS}
            \subsubsection{判断是否成环}
            拓扑排序形成的ans的sz != n 则成环
        \subsection{2SAT}
            \lstinputlisting{code/2sat.cc}
    \section{路径}
        \subsection{非递归欧拉回路}
            \lstinputlisting{code/eulerpath.cc}
    \section{匹配}
        \subsection{二分图最大匹配}
            最小点覆盖的点数 = 二分图最大匹配 \\
            最大独立集的点数 = 总点数 - 二分图最大匹配
        \lstinputlisting{code/erfentu.cc}
        \subsection{二分图最优匹配}
            \lstinputlisting{code/km.cc}
    \section{树}
        \subsection{prim}
            \lstinputlisting{code/prim.cc}
        \subsection{曼哈顿最小距离生成树}
            \lstinputlisting{code/manhadun.cc}
    \section{网络流}
        \subsection{最大流 Dinic}
            \lstinputlisting{code/dinic.cc}
        \subsection{最大流 ISAP}
            \lstinputlisting{code/isap.cc}
        \subsection{费用流}
            \lstinputlisting{code/mcmf.cc}
        \subsection{无源无汇有容量下界网络的可行流}
            \lstinputlisting{code/yy/wuyuan.cc}
    \section{其他}
        \subsection{层次遍历}
            每次记录队列当前的sz 然后在一次循环中只出队sz次
        \subsection{图解序列}
            图解序列:一系列非负整数可以构成一个简单图的度序列 \\
            Havel定理:对于n>1,长度为n的整数序列是图解序列当且仅当d'是图解序列,d'是的删除d中最大元素Δ并将紧跟的Δ个元素减1后序列
\chapter{计算几何}
    \section{判断凸包}
    \lstinputlisting{code/panduan.cc}
    \section{判断线段是否相交}
    \lstinputlisting{code/panduanin.cc}
    \section{求对称点 求交点}
    \lstinputlisting{code/qiuchuichengdian.cc}
    \section{终极模板}
    \lstinputlisting{code/jisuan.cc}
    \section{K次圆}
    \lstinputlisting{code/ko.cc}
\chapter{数据结构}
    \section{手写堆}
        \lstinputlisting{code/heap.cc}
    \section{左偏树}
        \lstinputlisting{code/zuopianshu.cc}
    \section{两优先队列模拟堆}
        \lstinputlisting{code/twopq.cc}
    \section{线段树}
        \lstinputlisting{code/xds.cc}
    \section{二维线段树}
        \lstinputlisting{code/2dst.cc}
    \section{Treap}
        \lstinputlisting{code/treap.cc}
    \section{splay}
        \lstinputlisting{code/splay.cc}
    \section{倍增LCA}
        \lstinputlisting{code/lca.cc}
    \section{主席树}
        \lstinputlisting{code/hjt.cc}
    \section{树剖}
        \lstinputlisting{code/shupo.cc}
    \section{点分治}
        \lstinputlisting{code/dianfenzhi.cc}
    \section{RMQ}
        \lstinputlisting{code/rmq.cc}
    \section{整体二分}
        \lstinputlisting{code/zhentierfen.cc}
    \section{莫队}
        \lstinputlisting{code/modui.cc}
    \section{KDtree}
        给N个K维点,找这K维点里面最近的M个点
        \lstinputlisting{code/kdtree.cc}
\chapter{字符串}
    \section{最小表示法}
        \lstinputlisting{code/minexp.cc}
    \section{KMP}
        \lstinputlisting{code/kmp.cc}
    \section{Manacher}
        \lstinputlisting{code/manacher.cc}
    \section{AC自动机}
        \lstinputlisting{code/ac.cc}
    \section{后缀数组}
        \lstinputlisting{code/sa.cc}
    \section{后缀自动机}
        \lstinputlisting{code/sam.cc}
\chapter{其他}
    \section{蔡勒公式}
        \lstinputlisting{code/caile.cc}
    \section{斜率DP}
        \lstinputlisting{code/xielvdp.cc}
    \section{最长子序列}
        \lstinputlisting{code/lis.cc}
    \section{四边形不等式}
        \lstinputlisting{code/sibianx.cc}
    \section{数位DP}
        \lstinputlisting{code/digdp.cc}
    \section{大数}
        \lstinputlisting{code/big.cc}
    \section{可以重复走的异或路径}
        \lstinputlisting{code/kyzfxor.cc}
\begin{thebibliography}{9}
\bibitem{交大模板} 
余勇. 
\textit{ACM国际大学生程序设计竞赛:算法与实现}. 
北京, 清华大学出版社, 2012
\end{thebibliography}
\end{document}

