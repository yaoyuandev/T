\documentclass[main.tex]{subfiles}
\begin{document}
\chapter{数据结构}
\section{手写堆}
    \lstinputlisting{code/heap.cc}
\section{左偏树}
    \lstinputlisting{code/zuopianshu.cc}
\section{两优先队列模拟堆}
    \lstinputlisting{code/twopq.cc}
\section{线段树}
    \lstinputlisting{code/xds.cc}
\section{二维线段树}
    \lstinputlisting{code/2dst.cc}
\section{Treap}
    \lstinputlisting{code/treap.cc}
\section{splay}
    \lstinputlisting{code/splay.cc}
\section{倍增LCA}
    \lstinputlisting{code/lca.cc}
\section{主席树}
    \lstinputlisting{code/hjt.cc}
\section{树剖}
    \lstinputlisting{code/shupo.cc}
\section{点分治}
    \lstinputlisting{code/dianfenzhi.cc}
\section{RMQ}
    \lstinputlisting{code/rmq.cc}
\section{整体二分}
    \lstinputlisting{code/zhentierfen.cc}
\section{莫队}
    \lstinputlisting{code/modui.cc}
\section{KDtree}
    给N个K维点,找这K维点里面最近的M个点
    \lstinputlisting{code/kdtree.cc}
\section{优先队列 可删除任意元素的堆}
    \lstinputlisting{code/ds/heap.cc}
\section{动态树带}
    \lstinputlisting{code/ds/dongtaitree.cc}
\section{动态树维护后缀自动机parent树}
    \lstinputlisting{code/ds/LCT.cc}
\section{虚树}
    \lstinputlisting{code/ds/xushu.cc}
\end{document}
